\section{Risiken}

Die Durchf\"uhrung vom Projekt beinhaltet mehrere Risiken, die teilweise w\"ahrend der bisherigen Analyse identifiziert wurden. Diesen Risiken wird mit Gegenmaßnahmen, soweit m\"oglich, in der weiteren Entwicklung entgegengewirkt.

\begin{table}[H] %set table here [H]
\begin{tabularx}{\textwidth}{ |X|X|X| }
	\hline  %table header
	\rowcolor[gray]{.8}%
	\rule{0pt}{18pt}%
	\textbf{Identifiziertes Risiko} & \textbf{Geplante Gegenmaßnahme} & \textbf{Risikoklasse} \\
	\hline  %table content
	Filteranlage vom VFB wird nicht zeitlich in Betrieb genommen &
	Testbetrieb auf \"offentliche Gelände & Gering \\
	\hline
	Unsachgemäßer Gebrauch der Hardware &
	Ersatz/Reparatur defekter Hardware & Gering \\
	\hline
	Server-Performance nicht ausreichend &
	Wechsel des Anbieters & Gering \\
	\hline
	TTN \"andert Schnittstellen &
	Quelltext anpassen & Gering \\
	\hline
	TTN l\"ost sich auf &
	Wechsel des LoRaWAN-Anbieters
	Anpassung des Quelltextes & Sehr gering \\
	\hline
\end{tabularx}
\end{table}