\documentclass[
11pt,
a4paper,
ngerman,
tablecaptionabove
]{article}

\usepackage{babel}	%übersetzt die englischen überschriften ins deutche
					% damit sind \tableofcontents und co gemeint

%\usepackage[ansinew]{inputenc}

\usepackage[utf8]{inputenc}
%\usepackege{xcolor}
\usepackage{mdframed}		%erzeugt rahmen "text box"
\usepackage{color,soul}		%zum setzten von hintergrund farben"
%\usepackage{lipsum}			%lorem ipsum generator
\usepackage{mathtools}		% mathematics
\usepackage{colortbl}
\usepackage{pdfpages}		% zum importieren von pdf-files
\usepackage{graphicx,color}	% zum importieren von grafiken
\usepackage{epstopdf}		% converting eps to PDF
\usepackage{float}			% für den text fluss um figures und tables
\usepackage{tabularx}

% modul wird nicht gefunden
%\usepackage{siunitx}		%si einheituen symbole

\title{Projektauftrag}
\author{Mark Otto \and }
\date{April 25, 2017}


\begin{document}
%%%%
% Titelseite
%%%%
\begin{titlepage}
\centering
\maketitle
\end{titlepage}

\tableofcontents
\listoffigures
\listoftables

\section{Versionsführung}
\begin{tabularx}{\textwidth}{ |X|X|X| }
  \hline
	\textbf{ Datum } & \textbf{ Bearbeiter } & \textbf{ Bemerkung } \\
  \hline 
  item 1  & item 2  & item 3 \\
  \hline
\end{tabularx}

\section{Projektziel}
Das Projekt „Wir haben noch keinen Namen“ entstand aus der Idee mithilfe von Messstationen umweltbezogene Daten im Berliner Stadtgebiet zu sammeln, aufzubereiten und schließlich als „Open Data“ zu veröffentlichen und sie somit auch Dritten für weitergehende Analysen und Visualisierungen zugänglich zu machen.
Ziel ist es nun die Messtationen erfolgreich in Betrieb zu nehmen, Daten zu sammeln, und diese Daten strukturiert in eine zentrale Datenbank abzulegen.
Da der Einsatz der Messstationen bisher nur ansatzweise erprobt und die genaue Reichweite noch nicht bekannt ist, muss sich das Sammeln der Daten ggf. auf ein Gebiet in Empfängernähe beschränken.

\section{Stakeholder}
Das Projekt wird für die Technologiestiftung Berlin durchgeführt. Zu den direkt Beteiligten gehören Dr. Christian Hammel, Dr. Benjamin Seibel und Sebastian Rauer.
Indirekt betroffen ist die Community „The Things Network“, da alle Daten an ihre Server gesendet werden und auch ihre Schnittstellen genutzt werden, um die Daten weiterzuverarbeiten.

\section{Umfang}

\section{Termine}

\subsection{Projektstrukturplan (als Anhang)}

\section{Budget}

\subsection{Kapazitätenplan (im Anhang)}

\section{Risiken}
 

\end{document}