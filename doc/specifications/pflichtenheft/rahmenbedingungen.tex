\section{Entwicklungs- und Produktionsrahmenbedingungen}

Die Messstationen bestehen im Kern aus einem Arduino Uno sowie dem Dragino-Shield. Daran angeschlossen werden die Sensoren zur Messung der Wasserqualität. Auf jeden Fall müssen ein pH-Wert, ein Wassertrübheits-, sowie ein Wassertemperatursensor vorhanden sein. Optional zur Bestimmung der Wasserqualität sind ein Leitfähigkeitssensor und ein Sensor zur Bestimmung der Menge des im Wasser gelösten Sauerstoffs. Auch wird ein geeignetes Gehäuse für die Station benötigt.
Aufgrund der mangelhaften Funknetzstärke an dem Messorten muss noch mind. Ein TTN Gateway installiert werden. Der geplante Standort ist der Versuchsfilter des Vereins Flussbad Berlin. Die benötigte Hardware wird duch die Technologiestiftung Berlin gestellt.
Softwareseitig wird eine Komponente zum Abgreifen der Messdaten aus dem TTN, ein Datenbanksystem für die Speicherung (Software auf dem Datenbankserver), eine Web-Applikation zum Anzeigen, sowie eine Schnittstelle (API) zum erweiterten Zugriff auf die Messdaten. Die Messinstrumente werden durch zu entwickelnde Software auf dem Arduino ausgewertet, diese Messdaten werden anschließend an TTN versendet.
Technisch begrenzt werden die Messstationen vom Energieverbrauch sowie der Datenrate bei TTN. Für das benutzte (freie) 868MHz Funkband besteht die Sendezeitbegrenzung von max. 1\% einer Stunde von Seiten der Bundesnetzagentur.
Finanzielle Grenzen sind dem Projekt von Seiten des Budgets der TSB gesetzt.

\subsection{Entwicklungsschritte}

\descriptionWhat{In welchen Schritten wird entwickelt/umgesetzt}

Die serverseitige Implementierung und der Aufbau einer Messstation sollen parallel durchgeführt werden. Des Weiteren muss ein Konto der TSB im TTN registriert werden. Als ein eigenes Konto von der Messstation wird eins von den Entwicklern verwendet.

\subsubsection{Server Implementierung}
\begin{itemize}
	\item Einrichtung eines Servers und einer Datenbank
	\item Implementiereung des Datenbankmodels
	\item Einpflegen der Daten in die Datenbank
	\item Definition einer Schnittstelle zwischen dem Server und Client
	\item Implementierung des Webservers
\end{itemize}

\subsubsection{Messstation}
\begin{itemize}
	\item Aufbau einer Messstation mit der Hardware
	\item Implementierung eines Programms auf der Messstation
	\item Test von Versenden der Messdaten und Datenerfassung im TSB-Konto
	\item Abdichten der Hardware-Komponente
\end{itemize}

\subsection{Entwicklungsergebnisse}

\descriptionWhat{Was sind die jeweiligen Ergebnisse der Entwicklungsschritte (Prototypenumfang, Fertigstellungsgrad etc.)}

\subsubsection{Server Implementierung}

Implementation des im Kapitel \ref{subsec:con_strct_comp} beschriebenen Datenbankmodells. Der Webserver hat eine Schnittstelle für die Daten die vom TTN kommen. Diese Schnittstelle Es werden nur POST-Request's verarbeitet. Es muss gewährleistet werden das der Client der den Request ausführt vom TTN kommt. Das TTN versendet die im JSON Format. Das heißt es werden nur Daten im JSON Format akzeptiert. GET-Request's auf der gleichen URL werden mit einem 404-error zurückgewiesen. Sämtliche Request's müssen mittels HTTPS erfolgen um sicherzustellen das es keine Datenmanipulation beim Einfügen der Daten gibt.
Der Webserver hat eine weitere Schnittstelle um die Daten Dritten bereitzustellen. Für das abrufen der Daten erfolgt mittels GET-Request's.

------------DRAFT----------vvv\\
Durch den Datenbankserver werden die Daten vom TTN aufgenommen und in eine Datenbank abgelegt. In der Datenbank sind die Daten mit der in Kapital \ref{subsec:con_strct_comp} geschriebenen Struktur gespeichert. 

Der Webserver sowie die Datenbank sind vom Anbieter 1\&1 gemietet. Der Webserver stellt Clients die Daten in der Form von JSON zur Verfügung.

\subsubsection{Messstation}
Aufbau einer Messstation mit der Hardware:
Eine Messstation ist mit den in Kapital \ref{subsec:func_require} geschriebenen Komponenten aufgebaut.

Implementierung eines Programms auf der Messstation:
Die Rohdaten der Sensoren werden durch den Arduino Uno soweit aufbereitet. Die Messdaten werden mittels des Dragino LoRa Shield versendet.

Test von Versenden der Messdaten und Datenerfassung im TSB-Konto:
TSB kann auf eine Applikation von einer Messstation zugreifen und ihre Daten in der TTN-Console lesen.

Abdichten der Hardware-Komponente:
Die einzelne Hardware-Komponente ist wasserdicht und die Messstation ist in einem wasserdichten Gehäuse installiert.

\subsection{Werkzeuge}

\descriptionWhat{Mit welchen Werkzeugen wird umgesetzt}

Zur Entwicklung werden die folgenden Werkzeuge verwendet.
M.001: etwas

Programm auf der Messstation: Arduino-IDE
Server:


Zur Modellierung eines Datenmodells wird XXX verwendet.
Zur Bezeichnung eines UML-Diagramms wird XXX verwendet.
