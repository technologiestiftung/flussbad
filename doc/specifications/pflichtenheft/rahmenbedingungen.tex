\section{Entwicklungs- und Produktionsrahmenbedingungen}

Die Messstationen bestehen im Kern aus einem Arduino Uno sowie dem Dragino-Shield. Daran angeschlossen werden die Sensoren zur Messung der Wasserqualität. Auf jeden Fall müssen ein pH-Wert, ein Wassertrübheits-, sowie ein Wassertemperatursensor vorhanden sein. Optional zur Bestimmung der Wasserqualität sind ein Leitfähigkeitssensor und ein Sensor zur Bestimmung der Menge des im Wasser gelösten Sauerstoffs. Auch wird ein geeignetes Gehäuse für die Station benötigt.\newline
Aufgrund der mangelhaften Funknetzstärke an dem Messorten muss noch mind. ein TTN Gateway installiert werden. Der geplante Standort ist der Versuchsfilter des Vereins Flussbad Berlin, die benötigte Hardware wird duch die Technologiestiftung Berlin gestellt. Softwareseitig wird eine Komponente zum Abgreifen der Messdaten aus dem TTN, ein Datenbanksystem für die Speicherung (Software auf dem Datenbankserver), eine Web-Applikation zum Anzeigen, sowie eine Schnittstelle (API) zum erweiterten Zugriff auf die Messdaten. Die Messinstrumente werden durch zu entwickelnde Software auf dem Arduino ausgewertet, diese Messdaten werden anschließend an TTN versendet.\newline
Technisch begrenzt werden die Messstationen vom Energieverbrauch sowie der Datenrate bei TTN. Für das benutzte (freie) 868MHz Funkband besteht die Sendezeitbegrenzung von max. 1\% einer Stunde von Seiten der Bundesnetzagentur.\newline
Finanzielle Grenzen sind dem Projekt von Seiten des Budgets der TSB gesetzt. Es existiert kein festes Budget, jedoch wird jede Ausgabe mit der TSB besprochen und gegebenenfalls von dieser genehmigt.

\subsection{Entwicklungsschritte}

Die serverseitige Implementierung und der Aufbau einer Messstation sollen parallel durchgeführt werden. Des Weiteren muss ein Konto der TSB im TTN registriert werden. Als ein eigenes Konto der Messstation wird eines der Entwickler verwendet.

\subsubsection{Server Implementierung}
\begin{itemize}
	\item Einrichtung eines Servers und einer Datenbank
	\item Implementierung des Datenbankmodells
	\item Einpflegen der Daten in die Datenbank
	\item Definition einer Schnittstelle zwischen dem Server und Client
	\item Implementierung des Webservers
\end{itemize}

\subsubsection{Messstation}
\begin{itemize}
	\item Aufbau einer Messstation mit der Hardware
	\item Implementierung eines Programms auf der Messstation
	\item Test von Versenden der Messdaten und Datenerfassung im TSB-Konto
	\item Abdichten der Hardware-Komponenten
\end{itemize}

\subsection{Entwicklungsergebnisse}

\subsubsection{Server Implementierung}

Implementation des im Kapitel \ref{subsec:con_strct_comp} beschriebenen Datenbankmodells. Der Webserver besitzt eine Schnittstelle für die Daten, die von TTN gesendet werden. Es muss gewährleistet sein, dass der Client, der die Anfrage stellt, tats\"achlich von TTN kommt. Da TTN seine Daten nur im JSON-Format zur Verf\"ugung stellt, werden von unserem Server nur Daten im JSON-Format akzeptiert. Ung\"ultige Anfragen werden mit einem Fehler zurückgewiesen. Sämtlicher Datenverkehr wird \"uber das HTTPS-Protokoll abgewickelt, um sicherzustellen, dass keine Manipulation der Daten w\"ahrend der \"Ubertragung stattfinden kann.\newline
Der Webserver stellt eine weitere Schnittstelle zur Verf\"ugung, um die Daten Dritten bereitzustellen. Das Abrufen der Daten erfolgt \"uber eine HTTP(S)-Anfrage durch Aufrufen einer URL. Die Antwort des Servers liefert Daten im JSON-Format. Durch den Datenbankserver werden die Daten vom TTN aufgenommen und in eine Datenbank abgelegt. In der Datenbank sind die Daten mit der in Kapital \ref{subsec:con_strct_comp} geschriebenen Struktur gespeichert.\newline
Der Webserver sowie die Datenbank sind vom Anbieter 1\&1 gemietet. Der Webserver stellt Clients (Dritten) die Daten in der Form von JSON zur Verfügung.

\subsubsection{Messstation}
Eine Messstation ist mit den in Kapital \ref{subsec:func_require} geschriebenen Komponenten aufgebaut. Die Rohdaten der Sensoren werden durch den Arduino Uno soweit aufbereitet, dass die versendeten Daten schon in von Menschen verständlicher Form sind. Die Messdaten werden mittels des Dragino LoRa Shield versendet. Die TSB kann auf eine Applikation von einer Messstation zugreifen und ihre Daten in der TTN-Console lesen. Die einzelnen Hardware-Komponenten sind wasserdicht und die Messstation ist in einem wasserdichten Gehäuse installiert.

\subsection{Werkzeuge}
\begin{itemize}
	\item[-] Arduino-IDE
	\item[-] MySQL Workbench
\end{itemize}
