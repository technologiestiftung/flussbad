\section{Systemschnittstellen}

Das System verf\"ugt \"uber mehrere Komponenten, die \"uber verschiedene
Schnittstellen miteinander kommunizieren.

\subsection{Messstation zu TTN}
Eine Messstation muss \"uber eine Weboberfl\"ache manuell in einer Applikation
eines TTN-Konto registriert werden. Ist dies erfolgt, werden die Daten \"uber
das LoRaWAN-Protokoll von der Messstation an TTN \"ubertragen.

\subsection{TTN zu Server}
TTN stellt die empfangenen Daten mithilfe sog. \glqq Integrations\grqq{} zur Verf\"ugung.
Eine \glqq Integration\grqq{} muss der jeweiligen Applikation, in der die Messstation registriert
wurde, hinzugef\"ugt werden.
In diesem Fall handelt es sich um eine \glqq HTTP-Integration\grqq. In dieser wird eine URL zu
einem Webserver hinterlegt, an den die von TTN empfangenen Daten im JSON-Format
weitergeleitet werden.

\subsection{Server zu \"Offentlichkeit}
Die gemessenen Daten werden Dritten \"uber eine HTTP(S)-Schnittstelle im JSON-Format zur Verf\"ugung gestellt.
Diese werden \"uber die Domain \url{http://www.berlinerdaten.de/} abrufbar sein.

\subsection{HW/SW-Konfiguration}

Im Quelltext der Software, die auf dem Arduino l\"auft, m\"ussen bestimmte, von TTN vorgegebene
Werte eingetragen werden. Hierzu z\"ahlen die zwei Schl\"ussel (\glqq Network Session Key\grqq,
\glqq Application Session Key\grqq), die zur Herstellung eines sicheren Kommunikationskanals
zwischen Messstation und TTN n\"otig sind, und die sog. \glqq Device Address\grqq, mit der
das jeweilige Endger\"at (hier Messstation) eindeutig identifiziert wird.
