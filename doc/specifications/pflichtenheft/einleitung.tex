\section{Einleitung}

\subsection{Dokumentverwaltung}

\subsubsection{Änderungshistorie}

\begin{table}[H] %set table here [H]
\begin{tabularx}{\textwidth}{ |X|X|X|X| }
	\hline  %table header
	\rowcolor[gray]{.8}%
	\rule{0pt}{18pt}%
	\textbf{Version} & \textbf{Datum} & \textbf{Autor} & \textbf{Änderungsgrund} \\
	\hline  %table content
	V 0.1 & xx.xx.xxxx & HR, MO, PA, KA & Initiale Version \\
	\hline
\end{tabularx}
\end{table}

\subsubsection{Abkürzungen}

\begin{table}[H] %set table here [H]
	\centering
	\begin{tabular}{ |l|l| } %set col alignment left [l] with vertical border
		\hline  %table header
		% set color for the header
		\rowcolor[gray]{.8}%
		\rule{0pt}{18pt}%
		\textbf{Abkürzung} & \textbf{Erläuterung} \\
		\hline  %table content
		LoRaWAN & Long Range Wide Area Network \\
		\hline
		TSB & Technologiestiftung Berlin \\
		\hline
		TTN & The Things Network \\
		\hline
	\end{tabular}
\end{table}

\subsubsection{Glossar}

\begin{table}[H]
	\begin{tabularx}{\textwidth}{ |l|X| }
		\hline  %table header
		\rowcolor[gray]{.8}%
		\rule{0pt}{18pt}%
		\textbf{Abkürzung} & \textbf{Erläuterung} \\
		\hline  %table content
		LoRaWAN & Ein energiesparendes Funkprotokoll, in Europa im 868MHz-Band. Es ermöglicht Endgeräten das Versenden von Daten über Funk an das Internet, indem die Daten durch Basisstationen aufgefangen und an Netzwerkserver weitergeleitet werden, auf welche über Schnittstellen aus dem Internet heraus zugegriffen werden kann. Auf umgekehrten Wege können auch Daten vom Internet heraus an die Endgeräte gesendet werden. \\
		\hline
		TTN & Ein auf LoRaWAN basierendes freies Netzwerk mit Community-betriebenen Basisstationen und Server. Es ermöglicht das einfache Versenden von Daten mittels LoRaWAN an Computer/Server im Internet. Hierzu leiten die Basisstationen die empfangenen Daten an die TTN-Server weiter, auf welche aus dem Internet aus über Schnittstellen zugegriffen werden kann. \\
		\hline
		Arduino Uno & Ein Mikrocontroller-Board auf Basis des ATmega328P. Durch seine Energiesparsamkeit, der vielen I/O-Pins, dem ADC (Analog-Digital-Konverter), sowie der Einfachheit der Programmierung eignet er sich gut als Controller für Messstationen, kleineren Steuerungen u.ä. \\
		\hline
		Dragino-Shield & Der vollständige Name lautet Dragino LoRa Shield. Dieses Shield für einen Arduino ermöglicht das Funken im 868MHz-Band, welches in der LoRaWAN-Technologie benutzt wird. Es ermöglicht uns das Versenden von Daten über TTN. \\
		\hline
		Messstation & Unsere Messstationen bestehen aus Sensoren zur Bestimmung der Wasserqualität, einem Arduino Uno, einem Dragino-Shield sowie einem Gehäuse. Der Arduino Uno ließt regelmäßig die Werte der Sensoren und verschickt sie mit Hilfe des Dragino-Shields über TTN. \\
		\hline
		Datenbankserver & Ein Server, auf welchem Software läuft, die dafür zuständig ist regelmäßig die Messdaten aus dem TTN zu laden und in eine Datenbank einzulagern. Der Lesezugriff durch Dritte auf die Daten wird ermöglicht. \\
		\hline
		Schutzklasse IP65 & Definiert durch die DIN~EN~60529. Gehäuse entsprechen der Schutzart IP65 wenn sie Staubdicht sind und Schutz gegen Strahlwasser (Düse) aus beliebiger Richtung bieten. \\
		\hline
		Node &  \\
		\hline
		Gateway & Eine Empfangstation, welche die von Nodes gesendeten Nachrichten empfängt und über das Internet an die Server von TTN weiterleitet. \\
		\hline
		PHP & PHP (rekursives Akronym für PHP: Hypertext Preprocessor) ist eine weit
			  verbreitete und für den allgemeinen Gebrauch bestimmte
			  Open Source-Skriptsprache, welche speziell für die Webprogrammierung
			  geeignet ist und in HTML eingebettet werden kann. \\
		\hline
		MySQL & MySQL ist eines der weltweit verbreitetsten relationalen
				Datenbankverwaltungssysteme. Es ist als Open-Source-Software sowie
				als kommerzielle Enterpriseversion für verschiedene Betriebssysteme
				verfügbar und bildet die Grundlage für viele dynamische Webauftritte. \\
		\hline
		JSON & Die JavaScript Object Notation, kurz JSON ist ein kompaktes Datenformat
			   in einer einfach lesbaren Textform zum Zweck des Datenaustauschs zwischen
			   Anwendungen. \\
		\hline
	\end{tabularx}
\end{table}

\subsubsection{Ablage des Dokuments}

\subsection{Projekthintergrund}

\subsubsection{Projektinitiierung und -zielsetzung}

\begin{itshape}
	Kurze Darstellung des Projektanlasses und Nennung des Projektziels.
	Nennung des Auftraggebers etc.
\end{itshape}

\paragraph{\uline{Anlass}}
Das Projekt \glqq Klare Sicht im Kupfergraben\grqq{} entstand aus der Idee der Technologiestiftung Berlin (TSB) mit Hilfe von Messstationen umweltbezogene Daten im Berliner Stadtgebiet zu sammeln, aufzubereiten und schließlich als \glqq Open Data\grqq{} zu veröffentlichen und sie somit auch Dritten für weitergehende Analysen und Visualisierungen zugänglich zu machen.
Die TSB vermittelt Wissen über digitale Chancen und Herausforderungen, entwickelt digitale Tools und gestaltet in gemeinsamen Projekten mit Stadtgesellschaft, Verwaltung und Unternehmen den digitalen Wandel in Berlin.

\paragraph{\uline{Zielsetzung}}
\commentKayo{Das Ziel muss \textbf{lösungsneutral} formuliert sein. Es darf keine Lösungswege vorwegnehmen: Es sollten weder Lösungen beschrieben noch bestimmte Lösung favorisiert werden.}%
Ziel des Projektes ist es Messstationen mit Sensoren auszustatten, erfolgreich in Betrieb zu nehmen, Daten zu sammeln, und diese anschließend zentral auf einen Datenbankserver abzulegen, um sie für weitere Verarbeitung verfügbar zu machen..

\subsubsection{Projekt-Meilensteine}

\input{}

\descriptionWhat{Darstellung der aktuellen Meilensteinplanung}

\subsection{Projektrahmenbedingungen}

\subsubsection{Organisation}

\descriptionWhat{ Projektteam \\ Kommunikationszyklen \\ Projektlaufwerk }

Das Projektteam besteht aus den folgenden Studenten der Technischen Informatik an der Beuth Hochschule für Technik Berlin.

\begin{table}[H]
	\centering
	\begin{tabular}{ |l|l| }
		\hline  %table header
		\rowcolor[gray]{.8}%
		\rule{0pt}{18pt}%
		\textbf{Matrikelnummer} & \textbf{Name, Vorname} \\
		\hline  %table content
		826054 & Abe, Kayoko \\
		\hline
		798168 & Albrecht, Philipp \\
		\hline
		830645 & Otto, Mark \\
		\hline
		835333 & Radde, Heiko \\
		\hline
	\end{tabular}
\end{table}

Die regelmäßige Kommunikation im Team findet via E-Mail, Messanger App und direkter Besprechung statt. Die Kommunikation mit dem Projektauftragsgeber TSB findet primär durch die Email und teilweise durch Treffen statt.
Als Projektlaufwerk wird Git verwendet.

\subsection{Referenzdokumente}

\descriptionWhat{ Nennung von Dokumenten und Ort der Aufbewahrung }

Es existieren bereits diverse Referenzprojekte, die verdeutlichen wie und wozu die über LoRaWAN gesammelten Daten verwendet werden können:

\begin{itemize}[noitemsep]
	\item Skopje Pulse (\url{http://skopjepulse.mk})
	\item Deutsche Bahn (\url{https://www.technologiestiftung-berlin.de/fileadmin/daten/media/publikationen/vortraege/materialien/1705050_Willner_Vortrag.pdf})
\end{itemize}

Als unser Messort stellt der Verein Flussbad Berlin seine Filteranlage im Spree zur Verfügung:

\begin{itemize}[noitemsep]
	\item Das Projekt Flussbad (\url{http://www.flussbad-berlin.de/projekt})
\end{itemize}

Die verfügbaren Gateways werden durch TTN Mapper gefunden:

\begin{itemize}[noitemsep]
	\item TTN Mapper (\url{https://ttnmapper.org})
\end{itemize}

\newpage


